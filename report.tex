% #######################################
% ########### FILL THESE IN #############
% #######################################
\def\mytitle{Tic-Tac-Toe Coursework Report}
\def\myauthor{Maria Luque Anguita}
\def\contact{40280156@napier.ac.uk}
\def\mymodule{Algorithms and Data Structures (SET08122)}
% #######################################
% #### YOU DON'T NEED TO TOUCH BELOW ####
% #######################################
\documentclass[10pt, a4paper]{article}
\usepackage[a4paper,outer=1.5cm,inner=1.5cm,top=1.75cm,bottom=1.5cm]{geometry}
\twocolumn
\usepackage{graphicx}
\graphicspath{{./images/}}
%colour our links, remove weird boxes
\usepackage[colorlinks,linkcolor={black},citecolor={blue!80!black},urlcolor={blue!80!black}]{hyperref}
%Stop indentation on new paragraphs
\usepackage[parfill]{parskip}
%% Arial-like font
\IfFileExists{uarial.sty}
{
    \usepackage[english]{babel}
    \usepackage[T1]{fontenc}
    \usepackage{uarial}
    \renewcommand{\familydefault}{\sfdefault}
}{
    \GenericError{}{Couldn't find Arial font}{ you may need to install 'nonfree' fonts on your system}{}
    \usepackage{lmodern}
    \renewcommand*\familydefault{\sfdefault}
}
%Napier logo top right
\usepackage{watermark}
%Lorem Ipusm dolor please don't leave any in you final report ;)
\usepackage{lipsum}
\usepackage{xcolor}
\usepackage{listings}
%give us the Capital H that we all know and love
\usepackage{float}
%tone down the line spacing after section titles
\usepackage{titlesec}
%Cool maths printing
\usepackage{amsmath}
%PseudoCode
\usepackage{algorithm2e}

\titlespacing{\subsection}{0pt}{\parskip}{-3pt}
\titlespacing{\subsubsection}{0pt}{\parskip}{-\parskip}
\titlespacing{\paragraph}{0pt}{\parskip}{\parskip}
\newcommand{\figuremacro}[5]{
    \begin{figure}[#1]
        \centering
        \includegraphics[width=#5\columnwidth]{#2}
        \caption[#3]{\textbf{#3}#4}
        \label{fig:#2}
    \end{figure}
}

\lstset{
	escapeinside={/*@}{@*/}, language=C++,
	basicstyle=\fontsize{8.5}{12}\selectfont,
	numbers=left,numbersep=2pt,xleftmargin=2pt,frame=tb,
    columns=fullflexible,showstringspaces=false,tabsize=4,
    keepspaces=true,showtabs=false,showspaces=false,
    backgroundcolor=\color{white}, morekeywords={inline,public,
    class,private,protected,struct},captionpos=t,lineskip=-0.4em,
	aboveskip=10pt, extendedchars=true, breaklines=true,
	prebreak = \raisebox{0ex}[0ex][0ex]{\ensuremath{\hookleftarrow}},
	keywordstyle=\color[rgb]{0,0,1},
	commentstyle=\color[rgb]{0.133,0.545,0.133},
	stringstyle=\color[rgb]{0.627,0.126,0.941}
}

\thiswatermark{\centering \put(336.5,-38.0){\includegraphics[scale=0.8]{logo}} }
\title{\mytitle}
\author{\myauthor\hspace{1em}\\\contact\\Edinburgh Napier University\hspace{0.5em}-\hspace{0.5em}\mymodule}
\date{}
\hypersetup{pdfauthor=\myauthor,pdftitle=\mytitle,pdfkeywords=\mykeywords}
\sloppy
% #######################################
% ########### START FROM HERE ###########
% #######################################
\begin{document}
    \maketitle

    \section{Introduction}

The aim of this coursework is to demonstrate my understanding of both theory and practise in relation to the content of the Algorithms and Data Structures module. The task is to implement a text-based Tic-Tac-Toe game using the C programming language paying special attention to the algorithms and structures used for it. For it I programmed a game where you can choose to play alone (versus the computer) or with another person, as well as saving your game to a file and continue playing from that file later on. The current situation of the game is stored in a char list and  moves and undone moves are stored in two different stacks.

	\section{Design}
explain how u designed and architected your software paying particular attention to the algorithms and data structures used.

As the book "An Introduction to Algorithms and Data Structures" states, lists are perhaps the most simple and most commonly used data structure. For that reason I chose to store the playing grid in a list of chars. A list is an ordered sequence of vertices where associated with each vertex is a data item, a previous vertex, and a next vertex, except the first one and last one who have null previous and next vertices~\cite{storer_2013}.

\begin{lstlisting}
    char numbers[9] = {'1', '2', '3', '4', '5', '6', '7', '8', '9'};
\end{lstlisting}

This list stores the numbers that basically show the user what the available moves are. Each number represents a square in the Tic Tac Toe grid. When a player makes a move, the selected number changes from the number to the player's mark (X or O)(Figure~\ref{fig:ttt2}).



For the undo and redo functions I had to keep track of all the moves played. For this I chose stacks. A stack is a list with only the operations of inserting and deleting data items at one end (the "top"). The first stack would store all of the original moves, and if the player chose to undo a move, the number at the top of the stack would go into the second stack, and so on. If the player wanted to redo the game, it was just a matter of doing the same thing but in the opposite way, from the second stack to the first "original-moves" stack. This way is easy, fast and efficient.

Pointers increase the memory needed for the list by a constant factor. If the data of the linked list are integers, the use of pointers doubles the space used by an array representation, but in exchange, allows for many operations to be donee in the same time.
Advantage: (disadvantage for stack)
- not necessary to choose a maximum list size in advance, as we do when we declare an array to hold a stack or a queue

	\section{Enhancements}
features that u would add or improve if u had more time

	\section{Critical evaluation}
exaplaining the features that u feel work well, or work poorly, and why you think this you should support your evaluation with experimental results.

    \section{Personal evaluation}
reflecting on what you learned, the challenges you faces, the methods you used to overcome challenges and how you feel you performed


\begin{lstlisting}[caption = Hello World! in c++]
#include <iostream>

int main() {
    std::cout << "Hello World!" << std::endl;
    std::cin.get();
    return 0;
}
\end{lstlisting}



\begin{algorithm}[h]
\For{$i = 0$ \KwTo $100$}{
 print\_number = true\;
\If{i is divisible by 3}{
 print "Fizz"\;
 print\_number = false\;
}
\If{i is divisible by 5}{
 print "Buzz"\;
 print\_number = false\;
}
\If{print\_number}{
    print i\;
}
print a newline\;
}
\caption{FizzBuzz}
\end{algorithm}







\bibliographystyle{ieeetr}
\bibliography{references}

\section{Appendix}
\begin{figure}[h!]
    \centering
    \includegraphics{images/ttt2.png}
    \caption{Caption}
    \label{fig:ttt2}
\end{figure}
\end{document}
